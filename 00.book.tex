\documentclass[pdftex,12pt, oneside]{article}

%\usepackage[paperwidth=8.5in, paperheight=13in]{geometry} % Folio
\usepackage[paperwidth=8.27in, paperheight=11.69in]{geometry} % A4

\usepackage{makeidx}         % allows index generation
\usepackage{graphicx}        % standard LaTeX graphics tool
                             % when including figure files
\usepackage[bottom]{footmisc}% places footnotes at page bottom
\usepackage[english]{babel}
\usepackage{enumerate}
\usepackage{paralist}
\usepackage{float}
\usepackage{gensymb}  
\usepackage{listings}
\usepackage{color}
\usepackage{mathtools} % atau \usepackage{amsmath}
\renewcommand{\baselinestretch}{1.5}

\newcommand{\HRule}{\rule{\linewidth}{0.5mm}}

\definecolor{codegreen}{rgb}{0,0.6,0}
\definecolor{codegray}{rgb}{0.5,0.5,0.5}
\definecolor{codepurple}{rgb}{0.58,0,0.82}
\definecolor{backcolor}{rgb}{0.95,0.95,0.92}

\lstdefinestyle{mystyle}{
  backgroundcolor=\color{backcolor},
  commentstyle=\color{codegreen},
  keywordstyle=\color{magenta},
  stringstyle=\color{codepurple},
  basicstyle=\footnotesize,
  breakatwhitespace=false,
  breaklines=true,
  captionpos=b,
  keepspaces=true,
  numbers=left,
  numbersep=5pt,
  showspaces=false,
  showstringspaces=false,
  showtabs=false,
  tabsize=2
}

\lstset{style=mystyle}


\begin{document}
\sloppy % biar section ga melebar melewati kertas

\begin{center}
{\large LAPORAN PELAKSANAAN UJI COBA PROGRAM - WS PBB}
\\[1cm]
xx Januari 2017\\
Priyanto Tamami, S.Kom.
\end{center}

%\frontmatter%%%%%%%%%%%%%%%%%%%%%%%%%%%%%%%%%%%%%%%%%%%%%%%%%%%%%%


%%%%%%%%%%%%%%%%%%%%%%%%%%%%%%%%%%%%%%%%%%%%%%%%%%%%%%%%%%%%%%%%%%%%%%

\section{PENDAHULUAN}

Pengujian program atau aplikasi yang dilakukan berupa \textit{unit testing} dan \textit{integration testing} menggunakan \textit{tools} yang sudah ada pada bahasa pemrograman Java yaitu JUnit.

JUnit ini nantinya akan melakukan testing per unit pada tiap fungsi / \textit{method} yang membangun sistem aplikasi sehingga diharapkan tiap unit menghasilkan keluaran yang diharapkan.

\section{OUTPUT PROGRAM}

Karena menggunakan JUnit, maka keluarannya akan menghasilkan informasi \texttt{fail} dan \texttt{passed}.

Kondisi pengujian seperti dijelaskan sebelumnya bahwa akan dilakukan dengan \textit{unit test} dan \textit{integration test}, rinciannya adalah sebagai berikut :

\begin{enumerate}[A.]
  \item \textit{Unit Test}
  
  \textit{Unit test} yang dilakukan akan dibagi berdasarkan kelas/objek yang terbentuk, berikut adalah nama \textit{file} yang melakukan \textit{unit test}, nama \textit{file} menunjukkan nama kelas/objek yang dilakukan \textit{unit test}.
  
  \begin{enumerate}[1.]
    \item RootControllerTest
    
    Pengujian kelas/objek \texttt{RootController} adalah untuk memastikan bahwa \textit{request} yang masuk ke \textit{server} mendapatkan respon yang diinginkan seperti nilai yang tertera pada basis data. Kode untuk melakukan pengujian pada kelas/objek \texttt{RootController} ini adalah sebagai berikut :
    
    \begin{lstlisting}
package lab.aikibo.controller;

import lab.aikibo.constant.StatusRespond;
import lab.aikibo.model.*;
import lab.aikibo.services.PembayaranServices;
import lab.aikibo.services.ReversalServices;
import lab.aikibo.services.SpptServices;
import org.joda.time.DateTime;
import org.junit.Before;
import org.junit.Test;
import org.junit.runner.RunWith;
import org.mockito.InjectMocks;
import org.mockito.Mock;
import org.mockito.Mockito;
import org.springframework.test.context.junit4.SpringRunner;

import javax.servlet.http.HttpServletRequest;
import java.math.BigInteger;

import static org.junit.Assert.assertEquals;
import static org.junit.Assert.assertNull;
import static org.mockito.Mockito.when;

/**
 * Created by tamami.
 */
@RunWith(SpringRunner.class)
public class RootControllerTest {

    @InjectMocks
    private RootController rootController = new RootController();

    HttpServletRequest mock = Mockito.mock(HttpServletRequest.class);

    @Mock
    private SpptServices spptServices;
    @Mock
    private PembayaranServices byrServices;
    @Mock
    private ReversalServices revServices;

    private StatusInq status;
    private StatusInq statusInqGagalDataTidakAda;
    private StatusInq statusInqError;
    private Sppt sppt;

    private StatusTrx statusTrxBerhasil;
    private StatusTrx statusTrxNihil;
    private StatusTrx statusTrxTerbayar;
    private StatusTrx statusTrxBatal;
    private StatusTrx statusTrxError;
    private StatusTrx statusTrxhnPajakBukanAngka;
    private StatusTrx statusTrxWaktuBayarLdWaktuCatat;
    private PembayaranSppt byrSppt;

    private StatusRev statusRevBerhasil;
    private StatusRev statusRevNihil;
    private StatusRev statusRevGanda;
    private StatusRev statusRevError;
    private ReversalPembayaran revSppt;

    @Before
    public void init() {
        sppt = new Sppt("332901000100100010", "2013", "FULAN", "BREBES",
                new BigInteger("10000"), new BigInteger("0"));
        status = new StatusInq(1, "Data Ditemukan", sppt);
        statusInqGagalDataTidakAda = new StatusInq(StatusRespond.DATA_INQ_NIHIL, "Data Tidak Ditemukan", null);
        statusInqError = new StatusInq(StatusRespond.DATABASE_ERROR, "Kesalahan DB", null);

        byrSppt = new PembayaranSppt("332901000100100010","2013","KODE_NTPD","4.1.1.12.1",
                new BigInteger("10000"), "4.1.1.12.2", new BigInteger("0"), "FULAN",
                "BREBES");
        statusTrxBerhasil = new StatusTrx(StatusRespond.APPROVED, "Pembayaran Telah Tercatat", byrSppt);
        statusTrxNihil = new StatusTrx(StatusRespond.TAGIHAN_TELAH_TERBAYAR,
                "Tagihan Telah Terbayar atau Pokok Pajak Nihil", null);
        statusTrxTerbayar = new StatusTrx(StatusRespond.TAGIHAN_TELAH_TERBAYAR,
                "Tagihan Telah Terbayar", null);
        statusTrxBatal = new StatusTrx(StatusRespond.JUMLAH_SETORAN_NIHIL,
                "Tagihan SPPT Telah Dibatalkan", null);
        statusTrxError = new StatusTrx(StatusRespond.DATABASE_ERROR,
                "Kesalahan Server", null);

        revSppt = new ReversalPembayaran("332901000100100010","2013","KODE_NTPD");
        statusRevBerhasil = new StatusRev(StatusRespond.APPROVED, "Proses Reversal Berhasil", revSppt);
        statusRevNihil = new StatusRev(StatusRespond.DATA_INQ_NIHIL, "Data Yang Diminta Tidak Ada", null);
        statusRevGanda = new StatusRev(StatusRespond.DATABASE_ERROR, "Data Tersebut Ganda", null);
        statusRevError = new StatusRev(StatusRespond.DATABASE_ERROR, "Kesalahan Server", null);
    }

    @Test
    public void testInquirySpptBerhasil() {
        when(spptServices.getSpptByNopThn("332901000100100010","2013",null))
                .thenReturn(status);

        assertEquals(1, rootController.getSppt("332901000100100010","2013", mock).getCode());
        assertEquals("Data Ditemukan",
                rootController.getSppt("332901000100100010", "2013", mock).getMessage());
        assertEquals("332901000100100010",
                rootController.getSppt("332901000100100010","2013", mock).getSppt().getNop());
        assertEquals("2013",
                rootController.getSppt("332901000100100010","2013", mock).getSppt().getThn());
        assertEquals("FULAN",
                rootController.getSppt("332901000100100010","2013", mock).getSppt().getNama());
        assertEquals("BREBES",
                rootController.getSppt("332901000100100010","2013", mock).getSppt().getAlamatOp());
        assertEquals(new BigInteger("10000"),
                rootController.getSppt("332901000100100010", "2013", mock).getSppt().getPokok());
        assertEquals(new BigInteger("0"),
                rootController.getSppt("332901000100100010", "2013", mock).getSppt().getDenda());
    }

    @Test
    public void testInquirySpptGagalKarnaNihil() {
        when(spptServices.getSpptByNopThn("332901000100100020","2013",null))
                .thenReturn(statusInqGagalDataTidakAda);

        assertEquals(10,
                rootController.getSppt("332901000100100020","2013", mock).getCode());
        assertEquals("Data Tidak Ditemukan",
                rootController.getSppt("332901000100100020","2013", mock).getMessage());
        assertNull(rootController.getSppt("332901000100100020","2013", mock).getSppt());
    }

    @Test
    public void testInquiryDbError() {
        when(spptServices.getSpptByNopThn("332901000100100030","2013",null))
                .thenReturn(statusInqError);

        assertEquals(4,
                rootController.getSppt("332901000100100030","2013", mock).getCode());
        assertEquals("Kesalahan DB",
                rootController.getSppt("332901000100100030","2013", mock).getMessage());
        assertNull(rootController.getSppt("332901000100100030","2013", mock).getSppt());
    }

    @Test
    public void testInquiryThnPajakBukanAngka() {
        assertEquals(36,
                rootController.getSppt("332901000100100010","2a13", mock).getCode());
        assertEquals("Tahun Pajak Mengandung Karakter Bukan Angka",
                rootController.getSppt("332901000100100010","2a13", mock).getMessage());
        assertNull(rootController.getSppt("332901000100100010","2a13", mock).getSppt());
    }

    @Test
    public void testTrxWaktuBayarLdWaktuCatat() {
        assertEquals(StatusRespond.TGL_JAM_BAYAR_LD_TGL_JAM_KIRIM,
                rootController.prosesPembayaran("332901000100100010","2013","22122017",
                        "1000", mock).getCode());
        assertEquals("Tanggal atau jam pada saat dibayarkan melebihi tanggal dan jam saat ini",
                rootController.prosesPembayaran("332901000100100010","2013","22122017",
                        "1000", mock).getMessage());
        assertNull(rootController.prosesPembayaran("332901000100100010","2013","22122017",
                "1000", mock).getByrSppt());
    }

    @Test
    public void testTrxSpptSukses() {
        when(byrServices.prosesPembayaran("332901000100100010","2013",
                new DateTime(2016, 12, 19, 10, 0),null))
                .thenReturn(statusTrxBerhasil);

        assertEquals(1,
                rootController.prosesPembayaran("332901000100100010","2013","19122016",
                        "1000", mock).getCode());
        assertEquals("Pembayaran Telah Tercatat",
                rootController.prosesPembayaran("332901000100100010","2013","19122016",
                        "1000", mock).getMessage());
        assertEquals("332901000100100010",
                rootController.prosesPembayaran("332901000100100010","2013","19122016",
                        "1000", mock).getByrSppt().getNop());
        assertEquals("2013",
                rootController.prosesPembayaran("332901000100100010","2013","19122016",
                        "1000", mock).getByrSppt().getThn());
        assertEquals("KODE_NTPD",
                rootController.prosesPembayaran("332901000100100010","2013","19122016",
                        "1000", mock).getByrSppt().getNtpd());
        assertEquals("4.1.1.12.1",
                rootController.prosesPembayaran("332901000100100010","2013","19122016",
                        "1000", mock).getByrSppt().getMataAnggaranPokok());
        assertEquals(new BigInteger("10000"),
                rootController.prosesPembayaran("332901000100100010","2013","19122016",
                        "1000", mock).getByrSppt().getPokok());
        assertEquals("4.1.1.12.2",
                rootController.prosesPembayaran("332901000100100010","2013","19122016",
                        "1000", mock).getByrSppt().getMataAnggaranSanksi());
        assertEquals(new BigInteger("0"),
                rootController.prosesPembayaran("332901000100100010", "2013","19122016",
                        "1000", mock).getByrSppt().getSanksi());
        assertEquals("FULAN",
                rootController.prosesPembayaran("332901000100100010", "2013", "19122016",
                        "1000", mock).getByrSppt().getNamaWp());
        assertEquals("BREBES",
                rootController.prosesPembayaran("332901000100100010","2013","19122016",
                        "1000", mock).getByrSppt().getAlamatOp());
    }



    @Test
    public void testTrxNihil() {
        when(byrServices.prosesPembayaran("332901000100100010","2013",
                new DateTime(2016, 12, 19, 10, 0),null))
                .thenReturn(statusTrxNihil);

        assertEquals(StatusRespond.TAGIHAN_TELAH_TERBAYAR,
                rootController.prosesPembayaran("332901000100100010","2013",
                        "19122016","1000", mock).getCode());
        assertEquals("Tagihan Telah Terbayar atau Pokok Pajak Nihil",
                rootController.prosesPembayaran("332901000100100010","2013","19122016",
                        "1000", mock).getMessage());
        assertNull(rootController.prosesPembayaran("332901000100100010","2013","19122016",
                "1000", mock).getByrSppt());
    }

    @Test
    public void testTrxTerbayar() {
        when(byrServices.prosesPembayaran("332901000100100010","2013",
                new DateTime(2016,12,19,10,0), null)).thenReturn(statusTrxTerbayar);

        assertEquals(StatusRespond.TAGIHAN_TELAH_TERBAYAR,
                rootController.prosesPembayaran("332901000100100010","2013","19122016",
                        "1000", mock).getCode());
        assertEquals("Tagihan Telah Terbayar",
                rootController.prosesPembayaran("332901000100100010","2013","19122016",
                        "1000", mock).getMessage());
        assertNull(rootController.prosesPembayaran("332901000100100010","2013","19122016",
                "1000", mock).getByrSppt());
    }

    @Test
    public void testTrxBatal() {
        when(byrServices.prosesPembayaran("332901000100100010","2013",
                new DateTime(2016,12,19,10,0), null)).thenReturn(statusTrxBatal);

        assertEquals(StatusRespond.JUMLAH_SETORAN_NIHIL,
                rootController.prosesPembayaran("332901000100100010","2013","19122016",
                        "1000", mock).getCode());
        assertEquals("Tagihan SPPT Telah Dibatalkan",
                rootController.prosesPembayaran("332901000100100010","2013","19122016",
                        "1000", mock).getMessage());
        assertNull(rootController.prosesPembayaran("332901000100100010","2013","19122016",
                "1000", mock).getByrSppt());
    }

    @Test
    public void testTrxError() {
        when(byrServices.prosesPembayaran("332901000100100010","2013",
                new DateTime(2016,12,19,10,0), null)).thenReturn(statusTrxError);

        assertEquals(StatusRespond.DATABASE_ERROR,
                rootController.prosesPembayaran("332901000100100010","2013","19122016",
                        "1000", mock).getCode());
        assertEquals("Kesalahan Server",
                rootController.prosesPembayaran("332901000100100010", "2013","19122016",
                        "1000", mock).getMessage());
        assertNull(rootController.prosesPembayaran("332901000100100010","2013","19122016",
                "1000", mock).getByrSppt());
    }

    @Test
    public void testRevSukses() {
        when(revServices.prosesReversal("332901000100100010","2013","KODE_NTPD", null))
                .thenReturn(statusRevBerhasil);

        assertEquals(StatusRespond.APPROVED,
                rootController.prosesReversal("332901000100100010","2013","KODE_NTPD", mock).getCode());
        assertEquals("Proses Reversal Berhasil",
                rootController.prosesReversal("332901000100100010","2013","KODE_NTPD", mock).getMessage());
        assertEquals("332901000100100010",
                rootController.prosesReversal("332901000100100010","2013","KODE_NTPD", mock)
                        .getRevPembayaran().getNop());
        assertEquals("2013",
                rootController.prosesReversal("332901000100100010","2013","KODE_NTPD", mock)
                        .getRevPembayaran().getThn());
        assertEquals("KODE_NTPD",
                rootController.prosesReversal("332901000100100010","2013","KODE_NTPD", mock)
                        .getRevPembayaran().getNtpd());
    }

    @Test
    public void testRevNihil() {
        when(revServices.prosesReversal("332901000100100010","2013","KODE_NTPD", null))
                .thenReturn(statusRevNihil);

        assertEquals(StatusRespond.DATA_INQ_NIHIL,
                rootController.prosesReversal("332901000100100010","2013","KODE_NTPD", mock).getCode());
        assertEquals("Data Yang Diminta Tidak Ada",
                rootController.prosesReversal("332901000100100010","2013","KODE_NTPD", mock).getMessage());
        assertNull(rootController.prosesReversal("332901000100100010","2013","KODE_NTPD", mock).
                getRevPembayaran());
    }

    @Test
    public void testRevGanda() {
        when(revServices.prosesReversal("332901000100100010","2013","KODE_NTPD", null))
                .thenReturn(statusRevGanda);

        assertEquals(StatusRespond.DATABASE_ERROR,
                rootController.prosesReversal("332901000100100010","2013","KODE_NTPD", mock).getCode());
        assertEquals("Data Tersebut Ganda",
                rootController.prosesReversal("332901000100100010","2013","KODE_NTPD", mock).getMessage());
        assertNull(rootController.prosesReversal("332901000100100010","2013","KODE_NTPD", mock)
                .getRevPembayaran());
    }

    @Test
    public void testRevError() {
        when(revServices.prosesReversal("332901000100100010","2013","KODE_NTPD", null))
                .thenReturn(statusRevError);

        assertEquals(StatusRespond.DATABASE_ERROR,
                rootController.prosesReversal("332901000100100010","2013","KODE_NTPD", mock).getCode());
        assertEquals("Kesalahan Server",
                rootController.prosesReversal("332901000100100010","2013","KODE_NTPD", mock).getMessage());
        assertNull(rootController.prosesReversal("332901000100100010","2013","KODE_NTPD", mock)
                .getRevPembayaran());
    }

}
    \end{lstlisting}
    
    Agar tidak mengganggu kondisi data pada sistem basis data yang berjalan, maka \textit{unit test} ini menggunakan data model sebagaimana disiapkan pada \textit{method} \texttt{init()}, kemudian datanya akan di \textit{mock} ke dalam \textit{service} yang menangani, seperti misalkan pada kasus \textit{inquiry} data SPPT, maka data model akan di \textit{mock} ke dalam objek \texttt{SpptServices}, yang apabila diujikan ada \textit{request} masuk melalui \textit{method} \texttt{getSppt}, maka seharusnya sistem akan menghasilkan data persis seperti apa yang telah di \textit{mock} ke dalam \textit{service}.
    
    \item StoreProceduresDaoImplTest
    \item PembayaranServicesImplTest
    \item ReversalServicesImplTest
    \item SpptServicesImplTest
  \end{enumerate}
  
  \item \textit{Integration Test}
  
  \textit{Integration test} akan melakukan tugasnya untuk menguji apakah koneksi dengan sistem basis data berjalan sebagaimana yang diharapkan, \textit{integration test} yang dilakukan hanya akan dibagi menjadi 2 (dua) bagian, yaitu :
  
  \begin{enumerate}[1.]
    \item HibernateConfigurationIT
    \item ServicesIT
  \end{enumerate}
\end{enumerate}


\section{KENDALA YANG DIHADAPI}

Kendala yang dihadapi pada saat melakukan \textit{unit test} tidak ada, hanya saja pada saat melakukan \textit{integration test}, dibutuhkan basis data model agar basis data aslinya tidak terpengaruh oleh kondisi \textit{test} yang merubah nilai dari isi basis data.


\section{KESALAHAN PROGRAM}

Kesalahan program yang ditemukan selama pengujian adalah pada saat memverifikasi kode mata anggaran untuk penerimaan pajak bumi dan bangunan, keputusannya apakah penerimaan pokok pajak dipisahkan mata anggarannya dengan denda administrasi pajak daerah.


\section{WAKTU PROSES UJI COBA}

Untuk proses uji cobanya sendiri sangat cepat, hanya kurang dari 1 jam, karena cukup melakukan eksekusi pada seluruh unit yang ada di dalam sistem aplikasi, dan melakukan eksekusi pada beberapa unit \textit{integration test}.


\end{document}